\documentclass[10pt, letterpaper]{article}
\pdfgentounicode=1

\usepackage[
    ignoreheadfoot,
    top=2cm, bottom=1.5cm, left=1.5cm, right=1.5cm, footskip=0cm
]{geometry}

\usepackage{titlesec, tabularx, array, enumitem, fontawesome5, amsmath, hyperref, paracol, fancyhdr, parskip, multicol}

\usepackage[scaled]{helvet}
\renewcommand{\familydefault}{\sfdefault}

\hypersetup{
    pdftitle={Abhav Vohra's CV},
    pdfauthor={Abhav Vohra},
    colorlinks=true,
    urlcolor=black,
    hidelinks
}

\pagestyle{empty}

\titleformat{\section}{\bfseries\large}{}{0em}{}[\titlerule]
\titlespacing{\section}{0pt}{0.3cm}{0.2cm}

\begin{document}

\begin{center}
    {\LARGE \textbf{ABHAV VOHRA}}\\[4pt]
    (347) 416-7797 \quad 
    voh.abhav@gmail.com \quad
    github.com/abhavvohra \quad
    linkedin.com/in/abhavvohra \quad
    abhavvohra.github.io \quad
    Brooklyn, NY 
\end{center}

\section{EDUCATION}
\textbf{M.S. in Computer Engineering,} New York University \hfill Sept 2022 – Dec 2024 \\
\textit{Relevant Coursework: Data Structures, Deep Learning, Reinforcement Learning, Advanced ML,} \hfill GPA: 3.67/4.0 \\
\textit{Large Language Models, Natural Language Processing, Probability \& Stochastic Processes} 
\\
\textbf{B.Tech in Electronics and Communication Engineering,} GGSIPU Delhi, India \hfill Aug 2017 – June 2021 



\section{WORK EXPERIENCE}

\textbf{AI Systems Engineer,} Easley Dunn Productions, Inc. (Remote, US) \hfill June 2025 - Present
\vspace{-0.5em}
\begin{itemize}[leftmargin=*, itemsep=0pt, topsep=0pt, partopsep=0pt, parsep=0pt]
\item Developed multi-agent deep reinforcement learning system for Monster Gridiron, a Unity-based American football game, implementing PPO (Proximal Policy Optimization) networks with Unity ML-Agents to enable intelligent offensive team control.
\item Implemented self-play training architecture enabling agents to learn coordinated offensive strategies through iterative competition against defensive opponents.
\end{itemize}

\textbf{AI Engineer (Intern),} Treevah LLC (Remote, US) \hfill Feb 2025 - May 2025 
\vspace{-0.5em}
\begin{itemize}[leftmargin=*, itemsep=0pt, topsep=0pt, partopsep=0pt, parsep=0pt]
\item Developed an intelligent file organization system leveraging OpenAI GPT-4o and LangGraph to create an automated workflow that analyzes uploaded text documents, extracts semantic content, and categorizes files based on content similarity.
\item Implemented human-in-the-loop validation with user confirmation prompts before file transfers to ensure accuracy and user control.
\end{itemize}


\textbf{ML Engineer,} Vitalth Forgers Pvt. Ltd. (New Delhi, India)  \hfill May 2022 - Aug 2022 
\vspace{-0.5em}
\begin{itemize}[leftmargin=*, itemsep=0pt, topsep=0pt, partopsep=0pt, parsep=0pt]
  \item Developed influenza detection system using fine-tuned transformer models on vital sign time-series data (temperature, Pulse, Respiratory Rate, SpO2), achieving 88\% accuracy.
  \item Implemented an LSTM network for influenza classification, processing sequential vital sign data to distinguish flu from other respiratory conditions with 18\% reduction in false positives.
\end{itemize}
\textbf{Software Engineer,} Antriksh Labs Pvt. Ltd. (Remote, India) \hfill June 2020 - Dec 2021 
\vspace{-0.5em}
\begin{itemize}[leftmargin=*, itemsep=0pt, topsep=0pt, partopsep=0pt, parsep=0pt]
\item Developed end-to-end anomaly detection pipeline using scikit-learn, and unsupervised learning to analyze sensor data from robotics equipment, predicting failures with 85\% accuracy and optimizing maintenance schedules.
\item Built time-series forecasting models with PyTorch for multi-sensor IoT data streams, creating real-time dashboards that achieved 30\% reduction in unplanned downtime across robotics automation systems.
\end{itemize}

\section{PROJECTS}

\textbf{Arxiv Research Agent} 
\vspace{-0.5em}
\begin{itemize}[leftmargin=*, itemsep=0pt, topsep=0pt, partopsep=0pt, parsep=0pt]
\item Developed an automated academic research system using LangGraph's multi-agent architecture, integrating ArXiv API with StateGraph orchestration to streamline scholarly paper retrieval, analysis, and synthesis for researchers and academics.
\item Built comprehensive workflow  that leverages reflection agents for iterative improvement, vector database integration for semantic search, and structured report generation with proper citations to enhance research productivity and accuracy.
\end{itemize}

\textbf{Multi-Agent Technical Documentation RAG System} 
\vspace{-0.5em}
\begin{itemize}[leftmargin=*, itemsep=0pt, topsep=0pt, partopsep=0pt, parsep=0pt]
\item Built a technical documentation assistant using AI agents (API, code generation, troubleshooting), processing open-source codebases with query routing and parallel execution to provide comprehensive developer solutions.
\item Deployed FastAPI service with ChromaDB vector database, semantic chunking strategies, and confidence scoring systems for multi-domain knowledge synthesis.
\end{itemize}

\section{TECHNICAL SKILLS}
\begin{tabularx}{\textwidth}{@{}lX@{}}
    \textbf{Programming Languages:} & Python , SQL , C, C++, C\#\\
    \textbf{ML/AI Frameworks:} & PyTorch , Scikit-learn, Hugging Face Transformers, SpaCy, XGBoost, OpenCV\\
    \textbf{AI/LLM Engineering:} & LangChain, LangGraph, LlamaIndex, CrewAI, Neo4j, RAG Systems, Graph RAG\\
    \textbf{MLOps \& Development:} & MLflow, Weights \& Biases, Gradio, FastAPI, Docker, Kubernetes, Jupyter, PySpark, AWS (SageMaker, EC2), Git, CI/CD (Jenkins/GitHub Actions), Airflow\\
    \textbf{Deep Learning \& Architecture:} & Transformers, Attention Mechanisms, Natural Language Processing (NLP), CNNs, GANs, VAEs, Diffusion Models, Graph Neural Networks,  Multi-modal Models, Large Language Models (LLMs), RAGs, Prompt Engineering\\
\end{tabularx}

\end{document}
